% CHANNEL theory vNext (LaTeX)
% CHArge aNd ioN NanoscaLe-to-device Link
% Generated by ChatGPT-5.2 Pro on 2026-01-28

\documentclass[11pt]{article}
\usepackage[a4paper,margin=1in]{geometry}
\usepackage{amsmath,amssymb,bm}
\usepackage{physics}
\usepackage{siunitx}
\usepackage{hyperref}
\usepackage{mathtools}
\usepackage{enumitem}
\usepackage{booktabs}

\newcommand{\kb}{k_{\mathrm B}}
\newcommand{\be}{\beta}
\newcommand{\eps}{\varepsilon}
\newcommand{\epsr}{\varepsilon_{\mathrm r}}
\newcommand{\epsrres}{\varepsilon_{\mathrm r}^{\mathrm{res}}}
\newcommand{\epszero}{\varepsilon_0}
\newcommand{\ee}{e}

\title{CHANNEL Theory vNext: MD\,$\rightarrow$\,Kernels\,$\rightarrow$\,Continuum\,$\rightarrow$\,Device}
\author{\textbf{CHANNEL: CHArge aNd ioN NanoscaLe-to-device Link}}
\date{January 28, 2026}

\begin{document}
\maketitle

\begin{abstract}
This document describes a thermodynamically-auditable continuum framework for OMIEC/electrochemical systems, designed to ingest nanoscale information (MD/CG/experiments) as \,\emph{profiles/kernels}, produce continuum predictions via Poisson--Nernst--Planck (PNP) with material-specific closure, and optionally map to device observables (e.g., OECT drain current). 

vNext introduces a clean separation between the invariant ``skeleton'' (conservation laws, electrostatic bookkeeping, flux--force structure, boundary control, and audits) and the material-specific \emph{closure} (kernels and free-energy / chemical-potential models). Three closure modes are supported:
(A) analytic $\Omega$-based closure, (B) $\Omega$-based closure augmented by an extra energy-density kernel $\omega_{\mathrm{extra}}$, and (C) $\mu$-closure via per-species dimensionless excess potentials $\phi^{\mathrm{ex}}_i$.
\end{abstract}

\tableofcontents

\section{Goals and scope}
\subsection{Primary scientific goal}
Given a material class (e.g., OMIECs), device geometry, electrolyte choice, and a nanoscale description (MD/CG/experimental profiles), build a single workflow that:
\begin{enumerate}[leftmargin=2em]
  \item produces continuum predictions of steady and transient electrochemical response ($Q$--$V_G$, $C^*$, hysteresis, ionic profiles, doping profiles), and
  \item provides explicit audits for internal consistency (Gauss/Maxwell bookkeeping, conservation, convergence), enabling systematic model refinement.
\end{enumerate}

\subsection{What is invariant vs what is material-specific}
\paragraph{Invariant skeleton (must stay explicit, not black-boxed).}
\begin{enumerate}[leftmargin=2em]
  \item Conservation for each conserved species $i$:
  \begin{equation}
    \partial_t c_i + \nabla\cdot \bm J_i = R_i.
  \end{equation}
  \item Electrostatic coupling through free charge:
  \begin{equation}
    \rho_{\mathrm{free}} = \rho_{\mathrm{base}} + \sum_i q_i c_i + \rho_{\mathrm{redox}}.
  \end{equation}
  \item Explicit choice of electrical control: Dirichlet (voltage-controlled) vs Neumann (charge-controlled), related by Legendre transforms.
  \item Flux--force structure (Onsager/gradient flow template):
  \begin{equation}
    \bm J_i = - M_i c_i \grad \mu_i \quad (\text{with }M_i=D_i/(\kb T)).
  \end{equation}
  \item Reaction detailed-balance structure: forward/backward rates must reproduce the chosen equilibrium.
  \item Audits: Gauss/Maxwell bookkeeping, $d\Omega/dV_G$ relation when $\Omega$ exists, numerical convergence and positivity.
\end{enumerate}

\paragraph{Material-specific closure (allowed to be analytic, numerical, or learned).}
Examples include $\epsr(\cdot)$, accessible-volume factors $h_i(\cdot)$, insertion free energies $\Delta\mu^0_i(\cdot)$, diffusion $D_i(\cdot)$, and redox energetics/kinetics.

\section{Geometry, fields, and notation}
We describe a 1D slab of thickness $d$ with coordinate $z\in[0,d]$. The model is written so that extension to 2D/3D is a drop-in replacement of $\partial_z$ by $\grad$ and of 1D finite-volume by multidimensional discretizations.

\subsection{State variables}
\begin{itemize}[leftmargin=2em]
  \item $c_i(z,t)$: number density of mobile species $i$ (units: $\si{m^{-3}}$).
  \item $\psi(z,t)$: electrostatic potential (units: $\si{V}$).
  \item $\alpha(z,t)$: redox/doping site occupancy fraction (dimensionless), when applicable.
\end{itemize}

\subsection{Kernels / profiles (from MD/CG/experiment)}
All kernels can be (i) a 1D profile $f(z)$ or (ii) a smooth field $f(z;\alpha)$ (or more generally $f(z;\bm s)$ for an extended state $\bm s$).

\begin{itemize}[leftmargin=2em]
  \item $\epsr(z;\alpha)$: relative dielectric profile.
  \item $n_s(z)$: density of redox sites (units: $\si{m^{-3}}$).
  \item $\rho_{\mathrm{base}}(z)$: fixed/background free-charge density (units: $\si{C\,m^{-3}}$).
  \item $h_i(z;\alpha)$: accessible volume fraction for species $i$ (dimensionless).
  \item $\Delta\mu^0_i(z;\alpha)$: one-body insertion free energy offset (units: $\si{J}$).
  \item $\omega_{\mathrm{extra}}(z;\alpha)$: optional extra energy-density term (units: $\si{J\,m^{-3}}$).
  \item $\phi^{\mathrm{ex}}_i(z;\alpha)$: optional \emph{dimensionless} excess potential for $\mu$-closure.
\end{itemize}

\paragraph{Reference anchoring.}
All kernels must be anchored to a reservoir reference so that at a chosen reservoir location $z_{\mathrm{res}}$ one has (convention)
\begin{equation}
  \psi(z_{\mathrm{res}})=0,\quad \epsr(z_{\mathrm{res}})=\epsrres,\quad h_i(z_{\mathrm{res}})=1,\quad \Delta\mu^0_i(z_{\mathrm{res}})=0,\quad \phi^{\mathrm{ex}}_i(z_{\mathrm{res}})=0.
\end{equation}
This anchoring is essential for avoiding double counting and for interpreting $V_G$.

\section{Electrostatics: Dirichlet electric enthalpy}
We adopt a Dirichlet (voltage-controlled) formulation with boundary conditions
\begin{equation}
  \psi(0)=0,\qquad \psi(d)=V_G.
\end{equation}

For a local dielectric $\epsr(z;\alpha)$, Gauss/Poisson reads
\begin{equation}
  -\partial_z\left(\epszero\epsr(z;\alpha)\,\partial_z \psi\right) = \rho_{\mathrm{free}}(z).
\end{equation}

The Dirichlet electric enthalpy density is written as
\begin{equation}
  \omega_{\mathrm{field}} = -\frac12\epszero\epsr\,|\partial_z\psi|^2 + \psi\,\rho_{\mathrm{free}},
\end{equation}
which is convenient for consistent variational bookkeeping and avoids electrostatic double counting.

\section{Ion thermodynamics: $q_0$ protocol (double-counting control)}
A recurring ambiguity in MD-to-continuum pipelines is whether the ``excess'' insertion free energy already contains an accessible-volume/sieving factor. CHANNEL treats this as an explicit protocol choice.

\subsection{q0\_strategy A: explicit $h_i$ + conditional $\Delta\mu^0_i$}
In strategy A, $h_i$ is a separate kernel and $\Delta\mu^0_i$ is a conditional insertion free energy \emph{beyond} the accessible-volume effect.
Define
\begin{equation}
  U_i^{\mathrm{ex}}(z;\alpha) = U_i^{\mathrm{Born}}(z;\alpha) + \Delta\mu^0_i(z;\alpha),
\end{equation}
with Born contribution
\begin{equation}
  U_i^{\mathrm{Born}}(z;\alpha) = \frac{q_i^2}{8\pi\epszero r^{\mathrm{Born}}_i}\left(\frac{1}{\epsr(z;\alpha)}-\frac{1}{\epsrres}\right).
\end{equation}
The ideal entropy uses $h_i$ as an accessible-volume correction:
\begin{equation}
  \mu_i = \kb T\ln\frac{c_i}{h_i c_i^{\mathrm{res}}} + U_i^{\mathrm{ex}} + q_i\psi + \cdots
\end{equation}
where $c_i^{\mathrm{res}}$ is the reservoir density.

\subsection{q0\_strategy B: lumped $\Delta\mu^0_i$ (force $h_i\equiv 1$)}
In strategy B, the user provides a single $\Delta\mu^0_i$ that already includes the effect of sieving/accessible volume. CHANNEL enforces $h_i\equiv 1$ and the same definition
\begin{equation}
  U_i^{\mathrm{ex}} = U_i^{\mathrm{Born}} + \Delta\mu^0_i.
\end{equation}

\paragraph{Implementation note (CHANNEL enforcement).}
CHANNEL always uses the decomposition $U_i^{\mathrm{ex}}=U_i^{\mathrm{Born}}+\Delta\mu^0_i$, and $h_i$ enters only through the ideal term and the drift potential ($-\ln h_i$). This makes the double-counting protocol auditable.

\section{Closure modes A/B/C}
The continuum skeleton requires chemical potentials $\mu_i$ (or equivalent drift potentials) and, for $\Omega$-based closures, a grand-potential functional.

\subsection{Mode A: analytic $\Omega$-based closure (default OMIEC)}
A minimal per-volume grand-potential density is
\begin{equation}
  \omega = \omega_{\mathrm{field}} + \omega_{\mathrm{ion}} + \omega_{\mathrm{redox}}\;\;(+\;\omega_{\mathrm{extra}}\text{ in Mode B}).
\end{equation}
The ion part (for q0\_strategy A) is
\begin{equation}
  \omega_{\mathrm{ion}} = \sum_i \left[\kb T\,c_i\left(\ln\frac{c_i}{h_i c_i^{\mathrm{res}}}-1\right) + c_i U_i^{\mathrm{ex}}\right].
\end{equation}
Then
\begin{equation}
  \mu_i = \frac{\delta \Omega}{\delta c_i} = \kb T\ln\frac{c_i}{h_i c_i^{\mathrm{res}}} + U_i^{\mathrm{ex}} + q_i\psi.
\end{equation}

\subsection{Mode B: $\Omega$-based closure + $\omega_{\mathrm{extra}}$}
Mode B keeps Mode A but adds an extra energy density kernel $\omega_{\mathrm{extra}}(z;\alpha)$ (units: $\si{J\,m^{-3}}$):
\begin{equation}
  \Omega_{\mathrm{extra}} = \int_0^d \omega_{\mathrm{extra}}(z;\alpha)\,dz.
\end{equation}
This term modifies $\delta\Omega/\delta\alpha$ through its local derivative $\partial\omega_{\mathrm{extra}}/\partial\alpha$ and therefore affects redox equilibrium and kinetics, while preserving auditability (since $\Omega$ remains defined).

\subsection{Mode C: $\mu$-closure via $\phi^{\mathrm{ex}}_i$}
Mode C targets ``unknown electrochemistry'' where a global $\Omega$ may not be known. Instead, per species $i$ we provide a dimensionless excess potential kernel $\phi^{\mathrm{ex}}_i(z;\alpha)$ and define
\begin{equation}
  \frac{\mu_i}{\kb T} = \ln\frac{c_i}{c_i^{\mathrm{res}}} + \phi^{\mathrm{ex}}_i(z;\alpha) + \be q_i \psi.
\end{equation}
The drift potential used by the transport discretization is
\begin{equation}
  \Phi_i \equiv \frac{\mu_i^{\mathrm{(no\,ideal)}}}{\kb T} = \phi^{\mathrm{ex}}_i + \be q_i\psi.
\end{equation}
\paragraph{Audit implication.}
Since $\Omega$ is not defined by default in Mode C, Maxwell/energy audits involving $\Omega$ are not applicable. One should instead (i) run integrability checks in the space of state variables, (ii) optionally reconstruct an effective $\Omega$ over the validated domain, and (iii) distill back to Mode B/A when possible.

\section{Redox/doping site model}
When redox sites are present, we include an occupancy field $\alpha(z,t)\in(0,1)$ and a site density $n_s(z)$. A minimal local redox free-energy density is
\begin{equation}
\omega_{\mathrm{redox}} = n_s\left[\kb T\left(\alpha\ln\alpha + (1-\alpha)\ln(1-\alpha)\right) + \alpha\Delta G^0 - \kb T\,\alpha\ln(K_X c_X)\right],
\end{equation}
where $c_X$ is the counterion density and $K_X$ is an equilibrium constant (units chosen so that $K_X c_X$ is dimensionless).

\subsection{Equilibrium condition}
In the absence of additional feedback terms, local equilibrium implies
\begin{equation}
  \ln\frac{\alpha}{1-\alpha} = \ln(K_X c_X) - \be\left(\Delta G^0 + \sigma_P\ee\,\psi\right),
\end{equation}
where $\sigma_P$ is the signed number of elementary charges per occupied site.

\subsection{Including $\alpha$-feedback from kernels (Mode A/B)}
When kernels depend on $\alpha$ (e.g., $\epsr(z;\alpha)$, $h_i(z;\alpha)$, $U_i^{\mathrm{ex}}(z;\alpha)$), the stationarity condition becomes
\begin{equation}
  \kb T\ln\frac{\alpha}{1-\alpha} + \Delta G^0 - \kb T\ln(K_X c_X) + \sigma_P\ee\,\psi + F(z)=0,
\end{equation}
where the feedback term per site (units: J) is
\begin{equation}
  F(z) = \frac{1}{n_s}\sum_i \left[c_i\,\pdv{U_i^{\mathrm{ex}}}{\alpha} - \kb T\,c_i\,\pdv{}{\alpha}\ln h_i\right]
        -\frac{\epszero}{2n_s}\pdv{\epsr}{\alpha}\,|\partial_z\psi|^2
        +\frac{1}{n_s}\pdv{\omega_{\mathrm{extra}}}{\alpha}\quad(\text{Mode B}).
\end{equation}

\subsection{Kinetics with detailed balance}
A simple two-state kinetic model is
\begin{equation}
  \partial_t\alpha = k_{\mathrm{on}}\,c_X\,(1-\alpha) - k_{\mathrm{off}}\,\alpha.
\end{equation}
Detailed balance requires
\begin{equation}
  \frac{k_{\mathrm{on}}}{k_{\mathrm{off}}} = K_X\exp\left[-\be\left(\Delta G^0 + \sigma_P\ee\psi + F\right)\right],
\end{equation}
so that the kinetic steady state matches the equilibrium condition.

\section{Continuum dynamics: transport + Poisson}
\subsection{Transport (gradient-flow / Nernst--Planck)}
Using $\bm J_i=-M_i c_i \grad\mu_i$ with $M_i=D_i/(\kb T)$ yields the familiar Nernst--Planck form
\begin{equation}
  \bm J_i = -D_i\left(\grad c_i + c_i\grad\Phi_i\right),
\end{equation}
where $\Phi_i$ is the dimensionless drift potential:
\begin{equation}
  \Phi_i = \be\left(U_i^{\mathrm{ex}}+q_i\psi\right)-\ln h_i\quad\text{(Mode A/B)},
  \qquad
  \Phi_i = \phi_i^{\mathrm{ex}}+\be q_i\psi\quad\text{(Mode C)}.
\end{equation}

\subsection{Boundary conditions (default)}
CHANNEL's default 1D implementation uses:
\begin{itemize}[leftmargin=2em]
  \item Left boundary ($z=0$): Dirichlet reservoir for mobile ions $c_i(0)=c_i^{\mathrm{res}}$ and electric potential $\psi(0)=0$.
  \item Right boundary ($z=d$): no-flux for ions and Dirichlet potential $\psi(d)=V_G$.
\end{itemize}
Other boundary types (blocking, Butler--Volmer, surface capacitance) are intended as plugins in the broader framework.

\section{Numerical algorithms (1D reference solvers)}
\subsection{Algorithm 1: MD/CG/experiment $\rightarrow$ profiles/kernels}
(Implemented outside CHANNEL, typically in PILOTS.)
\begin{enumerate}[leftmargin=2em]
  \item Define geometry and binning along $z$; gather MD samples (possibly conditioned on $\alpha$ or other slow variables).
  \item Estimate $\epsr(z;\alpha)$ (e.g., from polarization response / fluctuation estimators), $h_i(z;\alpha)$ (accessible volume), and insertion kernels $\Delta\mu^0_i(z;\alpha)$.
  \item Enforce anchors and domain constraints: $\epsr>0$, $0<h_i\le 1$, reference values at $z_{\mathrm{res}}$.
  \item Smooth/interpolate to continuous profiles or $z$--$\alpha$ fields.
\end{enumerate}

\subsection{Algorithm 2: stationary (equilibrium) solver}
Given $V_G$, iterate to a self-consistent fixed point:
\begin{enumerate}[leftmargin=2em]
  \item (Kernels) Evaluate kernels at current $\alpha(z)$.
  \item (Ions) Update $c_i$ from local equilibrium:
    \begin{itemize}
      \item Mode A/B: $c_i=h_i c_i^{\mathrm{res}}\exp[-\be(U_i^{\mathrm{ex}}+q_i\psi)]$.
      \item Mode C: $c_i=c_i^{\mathrm{res}}\exp[-(\phi_i^{\mathrm{ex}}+\be q_i\psi)]$.
    \end{itemize}
    If explicit counterion coupling is enabled (Mode A/B), solve the local nonlinear condition induced by $\omega_{\mathrm{redox}}$.
  \item (Poisson) Solve Poisson with Dirichlet BC to update $\psi$.
  \item (Redox) If redox is enabled, update $\alpha$ either
    \begin{itemize}
      \item without feedback (closed-form logistic update), or
      \item with feedback by solving the nonlinear stationarity condition including $F(z)$.
    \end{itemize}
  \item Apply damping/mixing to stabilize convergence.
\end{enumerate}

\subsection{Algorithm 3: transient solver (operator splitting + SG flux)}
For each timestep $t_n\to t_{n+1}$:
\begin{enumerate}[leftmargin=2em]
  \item Update boundary voltage $V_G(t_{n+1})$ and solve Poisson for $\psi$ using the old charges (quasi-static field update).
  \item Reaction step: update $\alpha$ using kinetics with detailed balance. If enabled (Mode A/B), include $F(z)$.
  \item Transport step: update mobile $c_i$ using finite-volume discretization with Scharfetter--Gummel fluxes (guarantees positivity under mild conditions).
  \item Solve Poisson self-consistently with updated charges.
  \item Record observables.
\end{enumerate}

\section{Observables and device mapping}
\subsection{Charge and capacitance}
Two consistent calculations of gate charge per area are used:
\begin{equation}
  Q_{\mathrm{gate}} = -\epszero\epsr(d)\,\partial_z\psi\big|_{z=d},
  \qquad
  Q_{\mathrm{vol}} = \int_0^d \rho_{\mathrm{free}}(z)\,dz.
\end{equation}
Their mismatch is an audit of Poisson/discretization consistency.

An equilibrium differential capacitance can be estimated by finite differences:
\begin{equation}
  C^*_{\mathrm{eq}} \approx \frac{Q_{\mathrm{gate}}(V_G+\delta V)-Q_{\mathrm{gate}}(V_G-\delta V)}{2\delta V}.
\end{equation}

\subsection{OECT drain current (optional mapping)}
A minimal mapping (placeholder) relates the integrated doping fraction $\bar\alpha$ and gate charge to a channel conductance model $I_D(V_G)$.
This module is intentionally pluggable.

\section{Audits and verification}
\subsection{Always applicable}
\begin{itemize}[leftmargin=2em]
  \item Poisson bookkeeping: $Q_{\mathrm{gate}}$ vs $Q_{\mathrm{vol}}$.
  \item Positivity: $c_i\ge 0$, $\alpha\in[0,1]$.
  \item Grid/time-step convergence of $Q$--$V_G$, $C^*$, hysteresis metrics.
\end{itemize}

\subsection{$\Omega$-based audits (Modes A/B)}
When $\Omega$ exists, a Dirichlet Maxwell relation holds (sign convention depends on the chosen electric enthalpy):
\begin{equation}
  \frac{d\Omega_{\mathrm{eq}}}{dV_G} = Q_{\mathrm{gate}}.
\end{equation}
CHANNEL verifies this numerically by symmetric finite difference.

\subsection{Mode C audits (recommended)}
For Mode C, implement integrability / path-independence checks for $\mu$-closures on a validated domain, and when possible reconstruct an effective $\Omega$ and distill it to an analytic or energy-based form (Mode B/A).

\section{Extension roadmap (Level-0/1/2)}
\begin{itemize}[leftmargin=2em]
  \item \textbf{Level-0 (geometry upgrade):} 2D/3D device geometry while using 1D kernels along thickness.
  \item \textbf{Level-1 (state-function libraries):} kernels as functions of local state $(\phi,\alpha,\lambda,\dots)$ evaluated on 2D/3D grids.
  \item \textbf{Level-2 (explicit microstructure):} full spatially varying $n_s(\bm r)$, $\rho_{\mathrm{base}}(\bm r)$ and anisotropic transport.
\end{itemize}

\section*{Provenance}
This vNext LaTeX is aligned with the current CHANNEL C++ implementation (closure modes A/B/C, q0\_strategy A/B, Poisson + SG PNP, optional redox feedback, optional $\omega_{\mathrm{extra}}$ and $\phi^{\mathrm{ex}}$ kernels).

\end{document}
